\subsection{Prove Multiplication in ZK}
The relation $\mathcal{R}_{\textrm{mult}}$ describes two commitments: $C$ commits to values $a$ and $b$ and $V$ commits to their product.

\begin{equation*}
  \mathcal{R}_{\textrm{mult}} = \{ ((G_0, G_1, H \in \mathbb{G}), (V, C \in \mathbb{G})) , (a, b, s_C, s_V \in \mathbb{Z}_p))  | V = abG_0 + s_VH, C = aG_0 + bG_1 + s_CH\}
\end{equation*}

The following protocol is zero-knowledge argument of knowledge for $\mathcal{R}_{\textrm{mult}}$.

\begin{enumerate}
\item Prover draws $s_a, s_b, s_S, \tau_1, \tau_2$ sends
  \begin{align*}
    S &= s_aG_0 + s_b G_1 + s_S H\\
    T_1 &= (as_b + bs_a)G_0 + \tau_1H\\
    T_2 &= s_as_bG_0 + \tau_2H\\
  \end{align*}
  \item Verifier sends random challenge $x$
  \item Prover sends
  \begin{align*}
    l = l(x) &= a + s_ax\\
    r = r(x) &= b + s_bx\\
    \mu &= s_C + xs_S\\
    \tau_x &= s_V + x\tau_1 + x^2\tau_2
  \end{align*}
  \item Verifier checks
    \begin{align*}
      C + xS &= lG_0 + rG_1 + \mu H\\
      V + xT_1 + x^2T_2 &= lrG_0 + \tau_x H \\
    \end{align*}
\end{enumerate}

\subsubsection{Completeness}
\begin{align*}
  C + xS &= aG_0 + bG_1 + s_CH + x(s_aG_0 + s_b G_1 + s_SH)\\
         &= (a + xs_a)G_0 + (b + xs_b)G_1 + (s_C + xs_S)H\\
         &= lG_0 + rG_1 + \mu H\\
  \\
  V + xT_1 + x^2T_2 &= abG_0 + s_VH+ x(as_b + bs_a)G_0 + x\tau_1H + x^2s_as_bG_0 + x^2\tau_2H\\
                    &= (ab + x(as_b + bs_a) + x^2s_as_b)G_0 + (s_V + x\tau_1 + x^2\tau_2)H\\
                    &= lrG_0 + \tau_xH
\end{align*}

\subsubsection{Perfect Special Honest Verifier Zero Knowledge}

\paragraph{PHVZK Proof in Bulletproofs}
Since the protocol is similar to Bulletproofs range proofs, the proof provided in the Bulletproofs \cite{bp1} paper applies directly.
It builds a simulator as follows:
\begin{enumerate}
  \item The simulator draws random $l, r, \mu, \tau_x, T_2$. By assumption the verifier is honest, therefore the simulator can draw a uniformly random $x$.
  \item The simulator sets $S = \frac{1}{x}(lG_0 + rG_1 + \mu H - C)$
  \item The simulator sets $T_1 = \frac{1}{x}(lrG_0 + \tau_xH - V - x^2T_2$)
\end{enumerate}
The Bulletproofs paper then concludes that this simulator runs in polynomial time all elements in the proof are either independently randomly distributed or their relationship is fully defined by the verification equations.

\paragraph{Variants of the Protocol that are not PSHVZK}
The above proof may appear to have ``gaps'' since it does not give justifications for all the values occuring in the protocol.
For example, it's not obvious why removing $\pi_2$ from the protocol would invalidate PSHVZK and the proof.

Note that in the definition of PSHVZK (\cref{def:pshvzk}), in contrast to the simulator, the adversary is not restricted to polynomial time and comes up with the statement and the witness.
Hence, the distinguisher $\adv_2$ can solve the discrete logarithm problem and knows witnesses $a$ and $s_c$, for example.

We will now discuss whether variants of the protocol are still PSHVZK.
\begin{itemize}
  \item \emph{Variant A}: $s_a$ is removed from the protocol, i.e., the prover sends $l = l(x) = a$. This is not PSHVZK as it's impossible to build a simulator. The adversary knows witness $a$ and can therefore easily notice if $l$ obtained from the transcript is not equal to $a$. The simulator doesn't have access to the witness. In order to produce a transcript with an indistinguishable $l$ it would either have to guess $a$ or break the hiding property of the commitment.
  \item \emph{Variant B:} $s_S$ is removed from the protocol, i.e., the prover sends $\mu = s_C$. This is not PSHVZK for the same reason that variant A is not PSHVZK. The adversary knows witness $s_C$ and could compare it with $\mu$ to distinguish simulated from real transcripts.
  \item \emph{Variant C:} $\tau_2$ is removed from the protocol, i.e., the prover sends $T_2 = s_as_bG_0$. This is probably not PSHVZK as it seems impossible to build a simulator. The adversary, knowing $a$, $b$ and $x$, can compute $s_a$ and $s_b$ from polynomials $l(x)$ and $r(x)$.
    The adversary could then check whether $s_as_bG_0$ is equal to $T_2$ obtained from the transcript.
    For that to hold, it seems like the simulator would have to know witnesses $a$ and $b$.
\end{itemize}

\begin{theorem}
  Above protocol for $\mathcal{R}_{\textrm{mult}}$ has perfect special HVZK.
\end{theorem}
\begin{pproof}
  \pf
  \step{1}{It suffices to describe a PPT simulator that on input of the statement outputs a transcript $\pi = (x, l, r, \mu, \tau_x, S, T_1, T_2)$ such that the probability distribution $\Pr(\pi' = \pi)$ for random variable $\pi'$ is equal to the probability distribution of real transcript between a honest verifier and a prover with access to the witness.
           To simplify the notation, we write $\Pr(\pi' = \pi)$ as $\Pr(\pi)$.}
  \step{2}{ Let $\mathcal{S}$ be a simulator which draws $x, l, r, \mu, \tau_x, T_2$ uniformly at random and sets
    \begin{align*}
      S &:= \frac{1}{x}(lG_0 + rG_1 + \mu H - C)\\
      T_1 &:= \frac{1}{x}(lrG_0 + \tau_xH - V - x^2T_2)
    \end{align*}
    The simulator runs in polynomial time.}
  \step{3}{The distribution of transcript $\pi$ output by the simulator is
    \begin{align*}
        \Pr(\pi) &= \Pr(x, l, r, \mu, \tau_x, S, T_1, T_2)\\
                        &= \begin{cases} 0 \text{ if } S \ne \frac{1}{x}(lG_0 + rG_1 + \mu H - C) \text{ or } T_1 \ne \frac{1}{x}(lrG_0 + \tau_xH - V - x^2T_2) \\
                             \Pr(x)\Pr(l)\Pr(r)\Pr(\mu)\Pr(\tau_x)\Pr(T_2) \text{ otherwise}
                             \end{cases}
    \end{align*}
    where $\Pr(x),\Pr(l),\Pr(r),\Pr(\mu),\Pr(\tau_x),\Pr(T_2)$ are uniform distributions.
    }
  \step{4}{The distribution of transcript $\pi$ in an real interaction is
    \begin{align*}
      \Pr(\pi) &= \Pr(x) \Pr(l|x, s_a) \Pr(r|x, s_b) \Pr(\mu| x, s_S) \Pr(\tau_x| x, \tau_1, \tau_2) \Pr(S| s_a, s_b, s_S) \Pr(T_1| s_a, s_b, \tau_1)\\
      &\qquad \Pr(T_2| s_a, s_b, \tau_2) \Pr(s_a)\Pr(s_b)\Pr(s_S)\Pr(\tau_1 )\Pr(\tau_2)
    \end{align*}
    where $\Pr(x),\Pr(s_a),\Pr(s_b),\Pr(s_S),\Pr(\tau_1),\Pr(\tau_2)$ are uniform distributions and all conditional probability distributions are 0 if the defining equation holds and 1 otherwise. For example, $\Pr(l|x, s_a) = 0$ if $l\ne a + s_ax$.}
   \step{5}{We can rewrite the distribution of transcript $\pi$ in an real interaction as
    \begin{align*}
        \Pr(\pi) &= \Pr(x, l, r, \mu, \tau_x, S, T_1, T_2)\\
                        &= \begin{cases} 0 \text{ if } S \ne s_aG_0 + s_bG_1 x^{-1}(s_C - \mu)H \text{ or } T_1 \ne (as_b + bs_a)G_0 + \tau_1H \text{ for }\\ \qquad s_a := x^{-1}(l - a), s_b := x^{-1}(r - b), \tau_1 := - x^{-1}(s_V  + x^2\tau_2 - \tau_x), \tau_2 := \log_H(T_2) - s_as_b\log_H(G_0) \\
                             \Pr(x)\Pr(l)\Pr(r)\Pr(\mu)\Pr(\tau_x)\Pr(T_2) \text{ otherwise}
                             \end{cases}
    \end{align*}
    where $\Pr(x),\Pr(l),\Pr(r),\Pr(\mu),\Pr(\tau_x),\Pr(T_2)$ are uniform distributions.
   }
   \begin{pproof}
     \pf\ \step{5a}{Since $\Pr(l|x, s_a)\Pr(s_a) = \Pr(l,s_a|x) = \Pr(s_a|x, l) \Pr(l)$ and similarly for $r$, $\mu$, as well as $\Pr(\tau_x| x, \tau_1, \tau_2)\Pr(\tau_1 ) = \Pr(\tau_1| x, \tau_x, \tau_2)\Pr(\tau_x )$ and $\Pr(T_2| s_a, s_b, \tau_2)\Pr(\tau_2) = \Pr(\tau_2| s_a, s_b, T_2)\Pr(T_2)$,
       we have
       \begin{align*}
      \Pr(\pi) &= \Pr(x) \Pr(s_a|x, l) \Pr(s_b|x, r) \Pr(s_S| x, \mu) \Pr(\tau_1| x, \tau_x, \tau_2) \Pr(S| s_a, s_b, s_S) \Pr(T_1| s_a, s_b, \tau_1)\\
      &\qquad \Pr(\tau_2| s_a, s_b, T_2) \Pr(l)\Pr(r)\Pr(\mu)\Pr(\tau_x )\Pr(T_2)
       \end{align*}
       }
     \step{5b}{We can rewrite this to get rid of hidden random variables $s_a, s_b, s_S, \tau_1, \tau_2$ by substitution:
       \begin{align*}
      \Pr(\pi) &= \Pr(x) \Pr(S| x, l, r, \mu) \Pr(T_1| x, l, r, \tau_x, T_2) \Pr(l)\Pr(r)\Pr(\mu)\Pr(\tau_x )\Pr(T_2).
       \end{align*}
     }
     \step{5c}{$\Pr(\pi) = 0$ if and only if $\Pr(S| x, l, r, \mu) = 0$ or $\Pr(T_1| x, l, r, \tau_x, T_2) = 0$, otherwise both probabilities are 1.}
    \end{pproof}
  \step{6}{The distributions of the real transcript and simulated transcripts are the same.}
  \begin{pproof}
    \pf\ In the real transcript we have $S = \frac{1}{x}(lG_0 + rG_1 + s_CH - (aG_0 + bG_1 + \mu H)) = \frac{1}{x}(lG_0 + rG_1 + s_CH - C)$ since the statement is valid. Similarly
      \begin{align*}
        T_1 &= (as_b + bs_a)G_0 - x^{-1}(s_V  + x^2(\log_H(T_2) - s_as_b\log_H(G_0)) - \tau_x)H\\
            &= \frac{1}{x}((xas_b + xbs_a)G_0 + \tau_xH - s_VH - x^2T_2 - x^2s_as_bG_0)\\
            &= \frac{1}{x}((xas_b + xbs_a - x^2s_as_b)G_0 + \tau_xH - s_VH - x^2T_2)\\
            &= \frac{1}{x}((lr - ab)G_0 + \tau_xH - s_VH - x^2T_2)\\
            &= \frac{1}{x}(lr G_0 + \tau_xH - V - x^2T_2)\\
      \end{align*}
  \end{pproof}
\end{pproof}