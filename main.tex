\documentclass{article}
% General document formatting
\usepackage[margin=0.7in]{geometry}
\usepackage[parfill]{parskip}
\usepackage[utf8]{inputenc}
\usepackage{todonotes}
% Related to math
\usepackage{amsmath,amssymb,amsfonts,amsthm,hyperref,thmtools}
% for nested proofs

\usepackage{cleveref}
\usepackage{graphicx}
\graphicspath{ {./images/} }
\usepackage[
    adversary,
    sets,
    probability,
    asymptotics,
    lambda,
    oracles,
    primitives,
    operators
]{cryptocode}

\usepackage[backend=bibtex,style=numeric,sorting=none]{biblatex}
%%% Save space in references
\AtEveryBibitem{
  \ifentrytype{misc}{
    \clearfield{howpublished} % for eprint entries
  }{}
  \clearlist{location}
  \clearname{editor}
  \clearfield{pages}

  \ifentrytype{inproceedings}{ % Remove a lot for proceedings
    \clearfield{year}
    \clearlist{editor}
    \clearfield{volume}
    \clearlist{publisher}
    \clearfield{series}
    \clearfield{note}
    \clearfield{url}
  }{}
}
\addbibresource{bib.bib}


\newcommand{\ip}[2]{\left\langle #1, #2 \right\rangle}
\newtheorem{theorem}{Theorem}
\newcommand{\Z}{\mathbb{Z}}
\newcommand{\setup}{\mathcal{K}}
\newtheorem{definition}{Definition}[section]

\title{Little Crypto Notebook}
\begin{document}
\maketitle
\tableofcontents

\section{Proving Zero-Knowledge}
This section demonstrates a non-trivial proof of Zero-Knowledge that aims to be more rigorous than the proofs commonly encountered in the academic literature.
The motivation is the belief that following a detailed proof enhances understanding of Zero-Knowledge and helps building an intuition.
The toy protocol we will explore is a heavily simplified variant of Bulletproofs~\cite{bp1} range proofs.
In short, it proves that a committed value is the product of two other committed values.

\subsection{Zero Knowledge Arguments of Knowledge}

A zero-knowledge argument of knowledge consists of a non-interactive PPT Turing
machine $\setup$ which outputs a \emph{common random string} $\sigma$, and two interactive PPT Turing machines $\prover$ (prover) and $\verifier$ (verifier).
The prover and verifier interacting on inputs $x$ and $y$ will
produce a transcript $\pi$ and output a bit $b$ indicating whether the verifier accepts, which we write $\pi \gets
\ip{\prover(x)}{\verifier(y)} = b$.
For any $\sigma$, a value $w$ is a \emph{witness} for a \emph{statement} $x$ if it
satisfies the relation $(\sigma, x, w) \in \mathcal{R}$.

A zero-knowledge argument of knowledge must satisfy completeness, soundness, and zero-knowledge.
Completeness requires that the
prover be able to convince the verifier to accept $x$ with overwhelming probability
if $(\sigma, x, w) \in \mathcal{R}$.
Soundness requires that the prover fail with
overwhelming probability to convince the verifier to accept if $(\sigma, x, w) \notin
\mathcal{R}$.
The zero-knowledge property requires
that the verifier learns nothing about the witness from
interacting with an honest prover. This is formalized via the
existence of a simulator that is able to
construct an identically distributed proof without knowledge of the witness.

\subsection{Perfect Special Honest Verifier Zero Knowledge}
\begin{definition}[Perfect Special Honest Verifier Zero-Knowledge]
    \label{def:pshvzk}
    The protocol $(\setup, \prover, \verifier)$ has perfect \emph{Special Honest Verifier Zero-Knowledge~(PSHVZK)}
    if there exists a probabilistic polynomial time (PPT) simulator $\mathcal{S}$
    such that for all pairs of adversaries $(\adv_1,\adv_2)$,
    \begin{align*}
      &\prob{
        \begin{array}{c}
          (\sigma, x, w) \in \mathcal{R}\\
          \land \ \adv_2(\sigma, \pi) = 1
        \end{array}
        \middle| % \sigma \gets \setup(1^{\lambda}),
        \begin{array}{c}
            \sigma \gets \setup(\secparam);
            (x, w, \rho) \gets \adv_1(\sigma);\\
            \pi \gets \ip{\prover(\sigma, x, w)}{\verifier(\sigma, x; \rho)}
        \end{array}
      }\\
      &= \prob{
        \begin{array}{c}
          (\sigma, x, w) \in \mathcal{R}\\
          \land \ \adv_2(\sigma, \pi) = 1
        \end{array}
        \middle| % \sigma \gets \setup(1^{\lambda}),
        \begin{array}{c}
            \sigma \gets \setup(\secparam);
            (x, w, \rho) \gets \adv_1(\sigma);\\
            \pi \gets \mathcal{S}(x, \rho)
        \end{array}
      }.\hspace*{2em}
    \end{align*}
\end{definition}
It follows that a protocol is not SHVZK if there's $(\adv_1,\adv_2)$ which can distinguish between a transcript $\pi$ generated by real interaction and a simulated transcript (i.e. one that's produced by a simulator without access to the witness $w$).

\subsection{Prove Multiplication in ZK}
The relation $\mathcal{R}_{\textrm{mult}}$ describes two commitments: $C$ commits to values $a$ and $b$ and $V$ commits to their product.

\begin{equation*}
  \mathcal{R}_{\textrm{mult}} = \{ ((G_0, G_1, H \in \mathbb{G}), (V, C \in \mathbb{G})) , (a, b, s_C, s_V \in \mathbb{Z}_p))  | V = abG_0 + s_VH, C = aG_0 + bG_1 + s_CH\}
\end{equation*}

The following protocol is zero-knowledge argument of knowledge for $\mathcal{R}_{\textrm{mult}}$.

\begin{enumerate}
\item Prover draws $s_a, s_b, s_S, \tau_1, \tau_2$ sends
  \begin{align*}
    S &= s_aG_0 + s_b G_1 + s_S H\\
    T_1 &= (as_b + bs_a)G_0 + \tau_1H\\
    T_2 &= s_as_bG_0 + \tau_2H\\
  \end{align*}
  \item Verifier sends random challenge $x$
  \item Prover sends
  \begin{align*}
    l = l(x) &= a + s_ax\\
    r = r(x) &= b + s_bx\\
    \mu &= s_C + xs_S\\
    \tau_x &= s_V + x\tau_1 + x^2\tau_2
  \end{align*}
  \item Verifier checks
    \begin{align*}
      C + xS &= lG_0 + rG_1 + \mu H\\
      V + xT_1 + x^2T_2 &= lrG_0 + \tau_x H \\
    \end{align*}
\end{enumerate}

\subsubsection{Completeness}
\begin{align*}
  C + xS &= aG_0 + bG_1 + s_CH + x(s_aG_0 + s_b G_1 + s_SH)\\
         &= (a + xs_a)G_0 + (b + xs_b)G_1 + (s_C + xs_S)H\\
         &= lG_0 + rG_1 + \mu H\\
  \\
  V + xT_1 + x^2T_2 &= abG_0 + s_VH+ x(as_b + bs_a)G_0 + x\tau_1H + x^2s_as_bG_0 + x^2\tau_2H\\
                    &= (ab + x(as_b + bs_a) + x^2s_as_b)G_0 + (s_V + x\tau_1 + x^2\tau_2)H\\
                    &= lrG_0 + \tau_xH
\end{align*}

\subsubsection{Perfect Special Honest Verifier Zero Knowledge}

\paragraph{PHVZK Proof in Bulletproofs}
Since the protocol is similar to Bulletproofs range proofs, the proof provided in the Bulletproofs \cite{bp1} paper applies directly.
It builds a simulator as follows:
\begin{enumerate}
  \item The simulator draws random $l, r, \mu, \tau_x, T_2$. By assumption the verifier is honest, therefore the simulator can draw a uniformly random $x$.
  \item The simulator sets $S = \frac{1}{x}(lG_0 + rG_1 + \mu H - C)$
  \item The simulator sets $T_1 = \frac{1}{x}(lrG_0 + \tau_xH - V - x^2T_2$)
\end{enumerate}
The Bulletproofs paper then concludes that this simulator runs in polynomial time all elements in the proof are either independently randomly distributed or their relationship is fully defined by the verification equations.

\paragraph{Variants of the Protocol that are not PSHVZK}
The above proof may appear to have ``gaps'' since it does not give justifications for all the values occuring in the protocol.
For example, it's not obvious why removing $\pi_2$ from the protocol would invalidate PSHVZK and the proof.

Note that in the definition of PSHVZK (\cref{def:pshvzk}), in contrast to the simulator, the adversary is not restricted to polynomial time and comes up with the statement and the witness.
Hence, the distinguisher $\adv_2$ can solve the discrete logarithm problem and knows witnesses $a$ and $s_c$, for example.

We will now discuss whether variants of the protocol are still PSHVZK.
\begin{itemize}
  \item \emph{Variant A}: $s_a$ is removed from the protocol, i.e., the prover sends $l = l(x) = a$. This is not PSHVZK as it's impossible to build a simulator. The adversary knows witness $a$ and can therefore easily notice if $l$ obtained from the transcript is not equal to $a$. The simulator doesn't have access to the witness. In order to produce a transcript with an indistinguishable $l$ it would either have to guess $a$ or break the hiding property of the commitment.
  \item \emph{Variant B:} $s_S$ is removed from the protocol, i.e., the prover sends $\mu = s_C$. This is not PSHVZK for the same reason that variant A is not PSHVZK. The adversary knows witness $s_C$ and could compare it with $\mu$ to distinguish simulated from real transcripts.
  \item \emph{Variant C:} $\tau_2$ is removed from the protocol, i.e., the prover sends $T_2 = s_as_bG_0$. This is probably not PSHVZK as it seems impossible to build a simulator. The adversary, knowing $a$, $b$ and $x$, can compute $s_a$ and $s_b$ from polynomials $l(x)$ and $r(x)$.
    The adversary could then check whether $s_as_bG_0$ is equal to $T_2$ obtained from the transcript.
    For that to hold, it seems like the simulator would have to know witnesses $a$ and $b$.
\end{itemize}

\begin{theorem}
  Above protocol for $\mathcal{R}_{\textrm{mult}}$ has perfect special HVZK.
\end{theorem}
\begin{proof}
     $ $\par
  \begin{enumerate}
  \item It suffices to describe a PPT simulator that on input of the statement outputs a transcript $\pi = (x, l, r, \mu, \tau_x, S, T_1, T_2)$ such that the probability distribution $\Pr(\pi' = \pi)$ for random variable $\pi'$ is equal to the probability distribution of a real transcript between a honest verifier and a prover with access to the witness.
           To simplify the notation, we write $\Pr(\pi' = \pi)$ as $\Pr(\pi)$.
  \item Let $\mathcal{S}$ be a simulator which draws $x, l, r, \mu, \tau_x, T_2$ uniformly at random and sets
    \begin{align*}
      S &:= \frac{1}{x}(lG_0 + rG_1 + \mu H - C)\\
      T_1 &:= \frac{1}{x}(lrG_0 + \tau_xH - V - x^2T_2)
    \end{align*}
    The simulator runs in polynomial time.
  \item The distribution of transcript output by the simulator is
    \begin{align*}
        \Pr(\pi) &= \Pr(x, l, r, \mu, \tau_x, S, T_1, T_2)\\
                        &= \begin{cases} 0 \text{ if } S \ne \frac{1}{x}(lG_0 + rG_1 + \mu H - C) \text{ or } T_1 \ne \frac{1}{x}(lrG_0 + \tau_xH - V - x^2T_2) \\
                             \Pr(x)\Pr(l)\Pr(r)\Pr(\mu)\Pr(\tau_x)\Pr(T_2) \text{ otherwise}
                             \end{cases}
    \end{align*}
    where $\Pr(x),\Pr(l),\Pr(r),\Pr(\mu),\Pr(\tau_x),\Pr(T_2)$ are uniform distributions.
  \item The distribution of transcript in a real interaction is
    \begin{align*}
      \Pr(\pi) &= \Pr(x) \Pr(l|x, s_a) \Pr(r|x, s_b) \Pr(\mu| x, s_S) \Pr(\tau_x| x, \tau_1, \tau_2) \Pr(S| s_a, s_b, s_S) \Pr(T_1| s_a, s_b, \tau_1)\\
      &\qquad \Pr(T_2| s_a, s_b, \tau_2) \Pr(s_a)\Pr(s_b)\Pr(s_S)\Pr(\tau_1 )\Pr(\tau_2)
    \end{align*}
    where $\Pr(x),\Pr(s_a),\Pr(s_b),\Pr(s_S),\Pr(\tau_1),\Pr(\tau_2)$ are uniform distributions and all conditional probability distributions are 1 if the defining equation holds and 0 otherwise. For example, $\Pr(l|x, s_a) = 0$ if $l\ne a + s_ax$.
   \item We can rewrite the distribution of transcript in a real interaction as
    \begin{align*}
        \Pr(\pi) &= \Pr(x, l, r, \mu, \tau_x, S, T_1, T_2)\\
                        &= \begin{cases} 0 \text{ if } S \ne s_aG_0 + s_bG_1 x^{-1}(s_C - \mu)H \text{ or } T_1 \ne (as_b + bs_a)G_0 + \tau_1H \text{ for }\\ \qquad s_a := x^{-1}(l - a), s_b := x^{-1}(r - b), \tau_1 := - x^{-1}(s_V  + x^2\tau_2 - \tau_x), \tau_2 := \log_H(T_2) - s_as_b\log_H(G_0) \\
                             \Pr(x)\Pr(l)\Pr(r)\Pr(\mu)\Pr(\tau_x)\Pr(T_2) \text{ otherwise}
                             \end{cases}
    \end{align*}
    where $\Pr(x)$ is a uniform distribution since the verifier is honest and $\Pr(l),\Pr(r),\Pr(\mu),\Pr(\tau_x),\Pr(T_2)$ are uniform distributions.
   \begin{innerproof}
     $ $\par
     \begin{enumerate}
     \item Since $\Pr(l|x, s_a)\Pr(s_a) = \Pr(l,s_a|x) = \Pr(s_a|x, l) \Pr(l)$ and similarly for $r$, $\mu$, as well as $\Pr(\tau_x| x, \tau_1, \tau_2)\Pr(\tau_1 ) = \Pr(\tau_1| x, \tau_x, \tau_2)\Pr(\tau_x )$ and $\Pr(T_2| s_a, s_b, \tau_2)\Pr(\tau_2) = \Pr(\tau_2| s_a, s_b, T_2)\Pr(T_2)$,
       we have
       \begin{align*}
      \Pr(\pi) &= \Pr(x) \Pr(s_a|x, l) \Pr(s_b|x, r) \Pr(s_S| x, \mu) \Pr(\tau_1| x, \tau_x, \tau_2) \Pr(S| s_a, s_b, s_S) \Pr(T_1| s_a, s_b, \tau_1)\\
      &\qquad \Pr(\tau_2| s_a, s_b, T_2) \Pr(l)\Pr(r)\Pr(\mu)\Pr(\tau_x )\Pr(T_2)
       \end{align*}
     \item We can rewrite this to get rid of hidden random variables $s_a, s_b, s_S, \tau_1, \tau_2$ by substitution.
       For example $\Pr(s_a|x, l)\Pr(S| s_a, s_b, s_S) = \Pr(S| x, l, s_b, s_S)$.
       By repeatedly applying the substition, we obtain
       \begin{align*}
      \Pr(\pi) &= \Pr(x) \Pr(S| x, l, r, \mu) \Pr(T_1| x, l, r, \tau_x, T_2) \Pr(l)\Pr(r)\Pr(\mu)\Pr(\tau_x )\Pr(T_2).
       \end{align*}
     \item $\Pr(\pi) = 0$ if and only if $\Pr(S| x, l, r, \mu) = 0$ or $\Pr(T_1| x, l, r, \tau_x, T_2) = 0$, otherwise both probabilities are 1.
       \begin{align*}
       \Pr(S| x, l, r, \mu) = 0 \Leftrightarrow S &\ne s_aG_0 + s_bG_1 s_SH\\
       \Pr(T_1| x, l, r, \tau_x, T_2) = 0 \Leftrightarrow T_1 &\ne (as_b + bs_a)G_0 + \tau_1H\\
       \end{align*}
       We can substitute $s_a, s_b, s_S, \tau_1$ with expressions of $x, l, r, \mu, \tau_x, T_2$.
     \end{enumerate}
    \end{innerproof}
  \item The distributions of the real transcript and simulated transcripts are equal.
  \begin{innerproof}
    For the real transcript we have
    \begin{align*}
      \Pr(\pi) = 0 \Leftrightarrow S \ne s_aG_0 + s_bG_1 x^{-1}(s_C - \mu)H \text{ or } T_1 \ne (as_b + bs_a)G_0 + \tau_1H
      \end{align*}
      We can rewrite $S$ as
      \begin{align*}
        S = \frac{1}{x}(lG_0 + rG_1 + s_CH - (aG_0 + bG_1 + \mu H))
      \end{align*}
      and further as $S = \frac{1}{x}(lG_0 + rG_1 + s_CH - C)$ because the prover has a witness for the statement per the definition of PSHVZK.
      Similarly, we can rewrite $T_1$ as
      \begin{align*}
        T_1 &= (as_b + bs_a)G_0 - x^{-1}(s_V  + x^2(\log_H(T_2) - s_as_b\log_H(G_0)) - \tau_x)H\\
            &= \frac{1}{x}((xas_b + xbs_a)G_0 + \tau_xH - s_VH - x^2T_2 - x^2s_as_bG_0)\\
            &= \frac{1}{x}((xas_b + xbs_a - x^2s_as_b)G_0 + \tau_xH - s_VH - x^2T_2)\\
            &= \frac{1}{x}((lr - ab)G_0 + \tau_xH - s_VH - x^2T_2)\\
            &= \frac{1}{x}(lr G_0 + \tau_xH - V - x^2T_2).\\
      \end{align*}
      Therefore, both distributions are equal.
  \end{innerproof}
 \end{enumerate}
\end{proof}

\paragraph{Another attempt at explaining $\tau_2$}
Using above proof, we can now obtain another perspective for why we can't remove $\tau_2$ from the protocol and retain PSHVZK.
If we removed $\tau_2$, the distribution of transcript $\pi$ in a real interaction would be
\begin{align*}
    \Pr(\pi) &= \Pr(x) \Pr(l|x, s_a) \Pr(r|x, s_b) \Pr(\mu| x, s_S) \Pr(\tau_x| x, \tau_1) \Pr(S| s_a, s_b, s_S) \Pr(T_1| s_a, s_b, \tau_1)\\
      &\qquad \Pr(T_2| s_a, s_b) \Pr(s_a)\Pr(s_b)\Pr(s_S)\Pr(\tau_1 )
\end{align*}

which we can rewrite, using the same strategies as in above proof, as

\begin{align*}
      \Pr(\pi) &= \Pr(x) \Pr(s_a|x, l) \Pr(s_b|x, r) \Pr(s_S| x, \mu) \Pr(\tau_1| x, \tau_x) \Pr(S| s_a, s_b, s_S) \Pr(T_1| s_a, s_b, \tau_1)\\
      &\qquad \Pr(T_2 | s_a, s_b) \Pr(l)\Pr(r)\Pr(\mu)\Pr(\tau_x )\\
               &= \Pr(x) \Pr(S| x, l, r, \mu) \Pr(T_1| x, l, r, \tau_x) \Pr(T_2 | x, l, r) \Pr(l)\Pr(r)\Pr(\mu)\Pr(\tau_x)\\
               &= \begin{cases} 0 \text{ if } S \ne s_aG_0 + s_bG_1 x^{-1}(s_C - \mu)H \text{ or } T_1 \ne (as_b + bs_a)G_0 + \tau_1H \text{ or } T_2 \ne s_as_bG_0\text{ for }\\ \qquad s_a := x^{-1}(l - a), s_b := x^{-1}(r - b), \tau_1 := - x^{-1}(s_V  - \tau_x) \\
                             \Pr(x)\Pr(l)\Pr(r)\Pr(\mu)\Pr(\tau_x) \text{ otherwise}
                             \end{cases}\\
              &= \begin{cases} 0 \text{ if } S \ne \frac{1}{x}(lG_0 + rG_1 + s_CH - C) \text{ or } T_1 \ne \frac{1}{x}(lr G_0 + \tau_xH - V - x^2T_2) \text{ or } T_2 \ne x^{-2}(lr - lb - ar + ab)G_0\\
                             \Pr(x)\Pr(l)\Pr(r)\Pr(\mu)\Pr(\tau_x) \text{ otherwise}
                             \end{cases}.
    \end{align*}
We have that $x^{-2}(lr - lb - ar + ab)G_0 = x^{-2}(lr - lb - ar + ab)G_0$ and therefore, a simulator would have to output $T_2 = x^{-2}(lr - lb - ar + ab)G_0$, which seems impossible with the values provided to the PPT simulator.


\printbibliography
\end{document}
